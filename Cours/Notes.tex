% Options for packages loaded elsewhere
\PassOptionsToPackage{unicode}{hyperref}
\PassOptionsToPackage{hyphens}{url}
%
\documentclass[
  11pt,
]{article}
\usepackage[]{mathpazo}
\usepackage{amsmath}
\usepackage{ifxetex,ifluatex}
\ifnum 0\ifxetex 1\fi\ifluatex 1\fi=0 % if pdftex
  \usepackage[T1]{fontenc}
  \usepackage[utf8]{inputenc}
  \usepackage{textcomp} % provide euro and other symbols
  \usepackage{amssymb}
\else % if luatex or xetex
  \usepackage{unicode-math}
  \defaultfontfeatures{Scale=MatchLowercase}
  \defaultfontfeatures[\rmfamily]{Ligatures=TeX,Scale=1}
\fi
% Use upquote if available, for straight quotes in verbatim environments
\IfFileExists{upquote.sty}{\usepackage{upquote}}{}
\IfFileExists{microtype.sty}{% use microtype if available
  \usepackage[]{microtype}
  \UseMicrotypeSet[protrusion]{basicmath} % disable protrusion for tt fonts
}{}
\makeatletter
\@ifundefined{KOMAClassName}{% if non-KOMA class
  \IfFileExists{parskip.sty}{%
    \usepackage{parskip}
  }{% else
    \setlength{\parindent}{0pt}
    \setlength{\parskip}{6pt plus 2pt minus 1pt}}
}{% if KOMA class
  \KOMAoptions{parskip=half}}
\makeatother
\usepackage{xcolor}
\IfFileExists{xurl.sty}{\usepackage{xurl}}{} % add URL line breaks if available
\IfFileExists{bookmark.sty}{\usepackage{bookmark}}{\usepackage{hyperref}}
\hypersetup{
  pdftitle={Quantitative Portfolio Management},
  pdfauthor={Patrick Hénaff},
  hidelinks,
  pdfcreator={LaTeX via pandoc}}
\urlstyle{same} % disable monospaced font for URLs
\usepackage[margin=1in]{geometry}
\usepackage{graphicx}
\makeatletter
\def\maxwidth{\ifdim\Gin@nat@width>\linewidth\linewidth\else\Gin@nat@width\fi}
\def\maxheight{\ifdim\Gin@nat@height>\textheight\textheight\else\Gin@nat@height\fi}
\makeatother
% Scale images if necessary, so that they will not overflow the page
% margins by default, and it is still possible to overwrite the defaults
% using explicit options in \includegraphics[width, height, ...]{}
\setkeys{Gin}{width=\maxwidth,height=\maxheight,keepaspectratio}
% Set default figure placement to htbp
\makeatletter
\def\fps@figure{htbp}
\makeatother
\setlength{\emergencystretch}{3em} % prevent overfull lines
\providecommand{\tightlist}{%
  \setlength{\itemsep}{0pt}\setlength{\parskip}{0pt}}
\setcounter{secnumdepth}{-\maxdimen} % remove section numbering

\linespread{1.05}
\usepackage[utf8]{inputenc}
\usepackage{amsthm}
\usepackage{xfrac}
\ifluatex
  \usepackage{selnolig}  % disable illegal ligatures
\fi
\newlength{\cslhangindent}
\setlength{\cslhangindent}{1.5em}
\newlength{\csllabelwidth}
\setlength{\csllabelwidth}{3em}
\newenvironment{CSLReferences}[2] % #1 hanging-ident, #2 entry spacing
 {% don't indent paragraphs
  \setlength{\parindent}{0pt}
  % turn on hanging indent if param 1 is 1
  \ifodd #1 \everypar{\setlength{\hangindent}{\cslhangindent}}\ignorespaces\fi
  % set entry spacing
  \ifnum #2 > 0
  \setlength{\parskip}{#2\baselineskip}
  \fi
 }%
 {}
\usepackage{calc}
\newcommand{\CSLBlock}[1]{#1\hfill\break}
\newcommand{\CSLLeftMargin}[1]{\parbox[t]{\csllabelwidth}{#1}}
\newcommand{\CSLRightInline}[1]{\parbox[t]{\linewidth - \csllabelwidth}{#1}\break}
\newcommand{\CSLIndent}[1]{\hspace{\cslhangindent}#1}

\title{Quantitative Portfolio Management}
\author{Patrick Hénaff}
\date{Feb 2021}

\begin{document}
\maketitle

\newcommand{\ones}{\mathbf{1}}
\newcommand{\onesT}{\mathbf{1}^T}

\newtheorem{thm}{Theorem}

In this short note, we summarize the mathematical elements of the
classical portfolio theory of Markowitz and Trenor-Black.

\hypertarget{arithmetic-vs.-geometric-mean}{%
\section{Arithmetic vs.~Geometric
mean}\label{arithmetic-vs.-geometric-mean}}

Let \(r_A\) and \(r_G\) be, respectively, the arithmetic and geometric
means of a series of returns:

\begin{align*}
r_A &= \frac{1}{n} \sum_{k=1}^n r_k \\
r_G &= \prod_{k=1}^n (1+r_k)^{1/n} - 1
\end{align*}

and let \(V\) be the variance of \(r_k\). We show that the geometric
mean, which correctly represents the increase in wealth from an
investment, is lower than the arithmetic mean.

The MacLaurin series for \((1+x)^{1/n}\) is:

\[
(1+x)^\frac{1}{n} = 1 + \frac{1}{n} x + \frac{1-n}{n^2} \frac{x^2}{2} + o(x^2)
\]

\[
r_G \approx \prod_{k=1}^n \left(1 + \frac{1}{n} r_k + \frac{1-n}{n^2} \frac{r_k^2}{2} \right) - 1
\]

Developping the product and keeping terms of order 2,

\[
r_G \approx \frac{1}{n} \sum_k r_k + \frac{1}{n^2} \sum_{k \neq l} r_k r_l + \frac{1-n}{2n^2} r^2_k
\]

\begin{align}
r_G & \approx r_A - \frac{1}{2} \left[ \frac{1}{n} \sum_k r^2_k - \frac{1}{n^2} \left( \sum_k r^2_k + 2 \sum_{k \neq l} r_k r_l \right) \right] \\
& \approx r_A - \frac{1}{2} \left[ \frac{1}{n} \sum_k r^2_k - \left(  \frac{1}{n} \sum_k r_k  \right)^2 \right] \\
& \approx r_A - \frac{1}{2} V, \ \ V >= 0
\end{align}

\hypertarget{quadratic-programming}{%
\section{Quadratic Programming}\label{quadratic-programming}}

\hypertarget{qp-with-equality-constraints}{%
\subsection{QP with equality
constraints}\label{qp-with-equality-constraints}}

\[
\begin{aligned}
    \mbox{min} \ \ & \frac{1}{2} w^T \Sigma w  \\
    \mbox{s.t.} & \\
    & A^T w  = b
  \end{aligned}
\]

Lagrangian:

\[
L(w, \lambda) = \frac{1}{2} w^T \Sigma w - \lambda^T \left(A^Tw -b \right)
\]

First order conditions:

\[
\left\{
                \begin{aligned}
                \Sigma w - A \lambda &= 0 \\
                A^Tw & = b
                \end{aligned}
              \right.
\]

or,

\[
\begin{bmatrix} \Sigma & -A \\ A^T & 0 \end{bmatrix}  \left[ \begin{array}{c} w \\ \lambda \end{array} \right] = \left[ \begin{array}{c} 0 \\ b \end{array} \right] 
\]

\hypertarget{special-case-of-minimum-variance-problem}{%
\subsection{Special case of Minimum Variance
problem}\label{special-case-of-minimum-variance-problem}}

\[
A = \begin{bmatrix} 1  \\ \vdots  \\ 1\end{bmatrix} \ \ \ b = \mu^*
\]

Solution:

\[
w = \lambda \Sigma^{-1} A
\]

Normalize so that weights sum to 1:

\[
w = \frac{\Sigma^{-1} \mathbf{1}}{\mathbf{1}^T\Sigma^{-1}\mathbf{1}}
\]

\hypertarget{mean-variance-model-markowitz1952}{%
\section{Mean-Variance model (Markowitz,
1952)}\label{mean-variance-model-markowitz1952}}

\[
\begin{aligned}
    \mbox{min} \ \ & \frac{1}{2} w^T \Sigma w  \\
    \mbox{s.t.} & \\
    \mathbf{1}^Tw & = 1 \\
        R^Tw & = R_P
  \end{aligned}
\]

Lagrangian:

\[
L(w, \lambda_1, \lambda_2) = \frac{1}{2} w^T \Sigma w - \lambda_1(\mathbf{1}^Tw-1) -\lambda_2(R^Tw - R_P)
\]

Solution of first order conditions:

\begin{equation}
\left\{
                \begin{aligned}
                \Sigma w - \lambda_1 \mathbf{1} -\lambda_2 R  &= 0 \label{eq:eq-1} \\
                \mathbf{1}^T w &= 1 \\
                R^Tw & = R_P
                \end{aligned}
              \right.
\end{equation}

Determination of \(\lambda_1\) and \(\lambda_2\):

\[
w = \Sigma^{-1} (\lambda_1 \mathbf{1} + \lambda_2 R)
\]

Define:

\begin{align*}
                a & = \mathbf{1}^T\Sigma^{-1} \mathbf{1}\\
                b & = \mathbf{1}^T\Sigma^{-1} R \\
                c & = R^T \Sigma^{-1} R 
                \end{align*}

Substitute in (\ref{eq:eq-1}):

\begin{equation*}
\left\{
                \begin{aligned}
                \lambda_1 a + \lambda_2 b & = 1 \\
                \lambda_1 b + \lambda_2 c & = R_P
                \end{aligned}
              \right.
\end{equation*}

Solution:

\begin{align*}
                \lambda_1 & = \frac{c - b R_P}{\Delta} \\
                \lambda_2 & = \frac{a R_P - b}{\Delta} \\
                \Delta & = ac - b^2
                \end{align*}

Note that:

\begin{align*}
\sigma^2_P & = w^{*T} \Sigma w^* \\
& = w^{*T} \Sigma \left( \lambda_1 \Sigma^{-1} \mathbf{1}+ \lambda_2 \Sigma^{-1} R \right) \\
& = \lambda_1 + \lambda_2 R_P
\end{align*}

Two remarkable solutions:

\begin{itemize}
\tightlist
\item
  Minimum variance portfolio
\end{itemize}

\begin{align*}
\frac{\partial \sigma^2_P}{\partial R_P} & = 0 \implies \\
\frac{2 a R_P - 2b}{\Delta} & = 0 \implies \\
R_P & = \frac{b}{a} \\
\sigma^2_P &= \frac{1}{a} \\
\lambda_1 &=  \frac{1}{a} \\
\lambda_2 &=  0
\end{align*}

The weights of the minimum variance portfolio:

\begin{align*}
w_g & = \lambda_1 \Sigma^{-1} \mathbf{1}\\
& = \frac{\Sigma^{-1} \mathbf{1}}{\mathbf{1}^T\Sigma^{-1} \mathbf{1}}
\end{align*}

\begin{itemize}
\tightlist
\item
  \(\lambda_1 = 0\)
\end{itemize}

This second solution gives \(\lambda_2 = \frac{1}{b}\) and the optimal
weights:

\begin{align*}
w_d & = \lambda_2 \Sigma^{-1} R \\
& = \frac{\Sigma^{-1} R}{\mathbf{1}^T\Sigma^{-1} R}
\end{align*}

\begin{thm}\label{thm:mv-1}
Any MV optimal portfolio $w^*_P$ with expected excess return $R_P$ can be decomposed into two MV portfolios.

$$
w^*_P = A w_g + (1-A) w_d
$$
\end{thm}

\begin{proof}
Since $w_P$ is MV optimal,
\begin{align*}
w_P &= \lambda_1 \Sigma^{-1} \mathbf{1}+ \lambda_2 \Sigma^{-1} R \\
&= \lambda_1 a w_g + \lambda_2 b w_d
\end{align*}

One can verify that
$$
\lambda_1 a  + \lambda_2 b = 1
$$
\end{proof}

\hypertarget{mv-model-with-riskless-asset}{%
\subsection{MV model with riskless
asset}\label{mv-model-with-riskless-asset}}

The tangency portfolio corresponds to the point on the efficient
frontier where the slope of the tangent \(\frac{R_M - r_f}{\sigma_M}\)
is maximized, where:

\[
\frac{R_M - r_f}{\sigma_M} = \frac{w^T (R - R_f)}{\sqrt{w^T \Sigma w}}
\]

Noting that the slope is unchanged when the weights \(w\) are multiplied
by a constant, the tangency portfolio is found by solving the following
QP problem for an arbitrary \(R^*>R_f\):

\[
\begin{aligned}
    \mbox{min} \ \ & \frac{1}{2} w^T \Sigma w  \\
    \mbox{s.t.} & \\
        \tilde{R}^Tw & = R^*
  \end{aligned}
\]

with \(\tilde{R} = R-R_f\).

Lagrangian:

\[
L(w, \lambda) = \frac{1}{2} w^T \Sigma w - \lambda \left(\tilde{R}^Tw - R^* \right)
\]

Which yields:

\begin{equation}
w^* = \lambda^* \Sigma^{-1} \tilde{R}
\label{eq:w-star}
\end{equation}

Normalize so that the weights sum to 1:

\begin{equation}
w^* = \frac{\Sigma^{-1} \tilde{R}}{\mathbf{1}^T\Sigma^{-1} \tilde{R}}
\label{eq:sharpe}
\end{equation}

The corresponding expected excess return is given by:

\[
E(R^*_P) = \frac{\tilde{R}^T \Sigma^{-1} \tilde{R}}{\mathbf{1}^T\Sigma^{-1} \tilde{R}}
\]

\hypertarget{maximum-sharpe-ratio-for-two-risky-assets}{%
\subsection{Maximum Sharpe ratio for two risky
assets}\label{maximum-sharpe-ratio-for-two-risky-assets}}

Given two assets, A and M, the allocation that maximizes the Sharpe
ratio is given by:

\begin{equation}
w_A = \frac{R_A \sigma^2_M - R_M \sigma_A \sigma_M \rho_{AM}}{R_A \sigma^2_M + R_M \sigma^2_A - (R_A+R_M) \sigma_A \sigma_M \rho_{AM}}
\label{eq:wA}
\end{equation}

\begin{proof}
Use equation (\ref{eq:sharpe}) with 
\begin{equation}
\Sigma = \begin{bmatrix} \sigma^2_A & \rho \sigma_A \sigma_M \\ \rho \sigma_A \sigma_M & \sigma^2_M \end{bmatrix}
\label{eq:sigma}
\end{equation}

$$
\Sigma^{-1} = \frac{1}{(1-\rho^2) \sigma^2_A\sigma^2_M} \begin{bmatrix} \sigma^2_M & -\rho \sigma_A \sigma_M \\ -\rho \sigma_A \sigma_M & \sigma^2_A \end{bmatrix}
$$
\end{proof}

\hypertarget{treynor-black-model-treynor1973}{%
\section{Treynor-Black Model (Treynor \& Black,
1973)}\label{treynor-black-model-treynor1973}}

Assets excess return is modeled by a single factor model:

\[
R_i = \alpha_i + \beta_i R_M + e_i
\]

where \(\alpha_i\) is the idiosyncratic excess return of asset \(i\),
and \(e_i \sim N(0, \sigma^2_i)\) is the specific risk.

\hypertarget{calculation-of-the-active-portfolio}{%
\subsection{Calculation of the active
portfolio}\label{calculation-of-the-active-portfolio}}

The active portfolio is determined by the idiosyncratic excess return
and the specific risk of each asset.

The specific risks are assumed to be independent:

\[
\Sigma_A = \begin{bmatrix}
\sigma^2(e_1) & & \\
& \ddots & \\
& & \sigma^2(e_n) \end{bmatrix}
\]

Using equation (\ref{eq:sharpe}), we get:

\[
w_{Ai} = \frac{\sfrac{\alpha_i}{\sigma^2_i}}{\sum \sfrac{\alpha_i}{\sigma^2_i} }
\]

So that the active portfolio has an excess return and variance given by:
\begin{align*}
R_A &= \alpha_A + \beta_A R_M \\
\sigma^2_A &= \beta^2_A \sigma^2_M + \sigma^2(e_A)
\end{align*}

with \begin{align*}
\alpha_A &= \sum w_{Ai} \alpha_i \\
\beta_A &= \sum w_{Ai} \beta_i \\
\sigma^2(e_A) &= \sum w^2_{Ai} \sigma^2(e_i)
\end{align*}

\hypertarget{allocation-of-wealth-between-the-active-portfolio-and-the-market-portfolio}{%
\subsection{Allocation of wealth between the active portfolio and the
market
portfolio}\label{allocation-of-wealth-between-the-active-portfolio-and-the-market-portfolio}}

A fraction \(w_A\) of wealth is allocated to the active portfolio, and
the balance \((1-w_A)\) to the market portfolio so as to maximize the
Sharpe ratio of the global portfolio \(xA + (1-x)M\).

Using equation (\ref{eq:wA}) we get after some algebra:

\[
w_A = \frac{\alpha_A \sigma^2_M}{\alpha_A \sigma^2_M(1-\beta_A) + R_M \sigma^2(e_A)}
\]

\hypertarget{separability-of-the-sharpe-ratio-in-the-active-portfolio}{%
\subsection{Separability of the Sharpe ratio in the active
portfolio}\label{separability-of-the-sharpe-ratio-in-the-active-portfolio}}

The first order condition for the optimal active portfolio is:

\begin{equation}
w_A = \lambda_A \Sigma^{-1} \alpha
\label{eq:FOC-1}
\end{equation}

Substitute in the expression

\[
\alpha_A = w^T_A \alpha
\]

to get:

\begin{equation}
\frac{\alpha_A}{\lambda_A} = \alpha^T \Sigma^{-1} \alpha
\label{eq:sep-1}
\end{equation}

We next get an expression for \(\lambda_A\) in terms of known
quantities:

\begin{align*}
\sigma^2(e_A) &= w^T_A \Sigma w_A \\
&= \lambda^2_A \alpha^T \Sigma^{-1}\Sigma\Sigma^{-1} \alpha \\
&= \lambda^2_A \alpha^T \Sigma^{-1} \alpha
\end{align*}

Therefore,

\begin{align*}
\frac{\sigma^2(e_A)}{\lambda^2_A} &= \alpha^T \Sigma^{-1} \alpha \\
&= \frac{\alpha_A}{\lambda_A}
\end{align*}

Which yields:

\[
\lambda_A = \frac{\sigma^2(e_A)}{\alpha_A}
\]

Use this result in equation (\ref{eq:sep-1}) to get:

\begin{align*}
\frac{\alpha^2_A}{\sigma^2(e_A)} &= \alpha^T \Sigma^{-1} \alpha \\
&= \sum_i \frac{\alpha^2_i}{\sigma^2(e_i)}
\label{eq:sep-2}
\end{align*}

which shows that the square of the Sharpe ratio of the active portfolio
is the sum of the squares of the Sharpe ratios of the assets forming
that portfolio.

\hypertarget{the-treynor-black-model-in-the-notation-of-the-1973-paper-and-separability-of-the-sharpe-ratio-between-the-active-and-market-portfolios}{%
\subsection{The Treynor-Black model in the notation of the 1973 paper
and separability of the Sharpe ratio between the active and market
portfolios}\label{the-treynor-black-model-in-the-notation-of-the-1973-paper-and-separability-of-the-sharpe-ratio-between-the-active-and-market-portfolios}}

The investment universe is composed of \(n\) assets with asset-specific
excess return:

\begin{align}
r_i &= \alpha_i + \beta_i r_M + e_i \ \ i=1, \ldots, n \\
E(r_i) &= \alpha_i + \beta_i E(r_M) = \mu_i
\label{eq:treynor-black-1}
\end{align}

and of the market asset itself. Let \(w_i, i=1, \ldots, n\) be the
investment in the assets with asset-specific excess returns, and \(w_M\)
the investment in the market asset.

Treynor and Black restate this portfolio as an investment in \(n+1\)
assets, asset 1 to \(n\) being only exposed to the specific risk, and
the \(n+1\) asset being only exposed to the market risk:

\[
w_{n+1} = w_M + \sum_{i=1}^n \beta_i w_i
\]

Note that these \(n+1\) assets are independent. The mean and variance of
the portfolio are:

\begin{align}
E(r_P) &= \sum_{i=1}^{n+1} w_i E(r_i) = \mu_P \\
\sigma^2_P &= \sum_{i=1}^{n+1} w^2_i \sigma^2_i
\end{align}

As usual, maximize the Sharpe ratio by solving:

\[
\begin{aligned}
    \mbox{min} \ \ & \frac{1}{2} w^T \Sigma w  \\
    \mbox{s.t.} & \\
        \mu^Tw & = \mu_P
  \end{aligned}
\]

Keeping in mind that the assets are independent, the Lagrangian is:

\[
L(w, \lambda) = \sum_{i=1}^{n+1} w^2_i \sigma^2_i - 2\lambda \left( \sum_{i=1}^{n+1} w_i \mu_i - \mu_P \right)
\]

First order conditions for optimality yield:

\[
2 w_i \sigma^2_i - 2 \lambda \mu_i = 0 \ \ \ \ i=1, \ldots , n+1
\]

or,

\begin{equation}
w_i = \lambda \frac{\mu_i}{\sigma^2_i}
\label{eq:wi}
\end{equation}

Substitute in (\ref{eq:treynor-black-1}) to get:

\begin{align}
\mu_P &= \lambda \sum_{i=1}^{n+1} \sfrac{\mu^2_i}{\sigma^2_i} \label{eq:lambda-1} \\
\sigma^2_P &= \lambda^2 \sum_{i=1}^{n+1} \mu^2_i \sigma^2_i   \label{eq:lambda-2}
\end{align}

so that,

\[
\lambda = \frac{\sigma^2_P}{\mu_P}
\]

To summarize, the weights of the assets in the active portfolio are:

\[
w_i = \frac{\mu_i}{\mu_P} \frac{\sigma^2_P}{\sigma^2_i} \ \ \ i=1, \ldots, n
\]

To determine the investment in the market asset, \(w_M\), recall that,

\begin{align}
\mu_{n+1} &= E(r_M) = \mu_M \\
\sigma^2_{n+1} &= \sigma^2_M
\end{align}

Thus,

\begin{align}
w_{n+1} &= \sum_{i=1}^n w_i \beta_i + w_M \\
&= \lambda \frac{\mu_M}{\sigma^2_M}
\end{align}

From equation (\ref{eq:wi}, we have:

\[
\sum_{i=1}^n w_i \beta_i = \lambda \sum_{i=1}^n \frac{\beta_i \mu_i}{\sigma^2_i}
\]

So that the investment in the market asset can be written as

\[
w_M = \lambda \left[ \frac{\mu_M}{\sigma^2_M} - \sum_{i=1}^n \frac{\beta_i \mu_i}{\sigma^2_i} \right]
\]

To establish the separability of the Sharpe ratio between the active and
the market portfolios, combine equations (\ref{eq:lambda-1}) and
(\ref{eq:lambda-2}) to get:

\[
\frac{\mu^2_P}{\sigma^2_P} = \sum_{i=1}^{n+1} \frac{\mu^2_i}{\sigma^2_i}
\]

Denoting \(S_A, S_M, S_P\) the Sharpe ratios of, respectively, the
active, market and overall portfolios, we can restate the previous
equation as:

\begin{align}
S^2_P &= \sum_{i=1}^{n} \frac{\mu^2_i}{\sigma^2_i} + S^2_M \\
&= \frac{\alpha^2_A}{\sigma^2_A + S^2_M} \\
S^2_A + S^2_M
\end{align}

Treynor and Black call \(\alpha_A = \sum_{i=1}^n w_i \alpha_i\) the
``appraisal premium'' and \(\sigma^2_A = \sum_{i=1}^n w^2_i \sigma^2_i\)
the ``appraisal risk.''

\hypertarget{bibliography}{%
\section*{Bibliography}\label{bibliography}}
\addcontentsline{toc}{section}{Bibliography}

\hypertarget{refs}{}
\begin{CSLReferences}{1}{0}
\leavevmode\hypertarget{ref-Markowitz1952}{}%
Markowitz, H. M. (1952). {Portfolio Selection}. \emph{The Journal of
Finance}, \emph{7}(1), 77--91.

\leavevmode\hypertarget{ref-Treynor1973}{}%
Treynor, J. L., \& Black, F. (1973). {How to Use Security Analysis to
Improve Portfolio Selection}. \emph{The Journal of Business},
\emph{46}(1), 66--86. \url{http://www.jstor.org/stable/2351280}

\end{CSLReferences}

\end{document}
