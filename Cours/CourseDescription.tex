% Options for packages loaded elsewhere
\PassOptionsToPackage{unicode}{hyperref}
\PassOptionsToPackage{hyphens}{url}
%
\documentclass[
  11pt,
]{article}
\usepackage[]{mathpazo}
\usepackage{amsmath}
\usepackage{ifxetex,ifluatex}
\ifnum 0\ifxetex 1\fi\ifluatex 1\fi=0 % if pdftex
  \usepackage[T1]{fontenc}
  \usepackage[utf8]{inputenc}
  \usepackage{textcomp} % provide euro and other symbols
  \usepackage{amssymb}
\else % if luatex or xetex
  \usepackage{unicode-math}
  \defaultfontfeatures{Scale=MatchLowercase}
  \defaultfontfeatures[\rmfamily]{Ligatures=TeX,Scale=1}
\fi
% Use upquote if available, for straight quotes in verbatim environments
\IfFileExists{upquote.sty}{\usepackage{upquote}}{}
\IfFileExists{microtype.sty}{% use microtype if available
  \usepackage[]{microtype}
  \UseMicrotypeSet[protrusion]{basicmath} % disable protrusion for tt fonts
}{}
\makeatletter
\@ifundefined{KOMAClassName}{% if non-KOMA class
  \IfFileExists{parskip.sty}{%
    \usepackage{parskip}
  }{% else
    \setlength{\parindent}{0pt}
    \setlength{\parskip}{6pt plus 2pt minus 1pt}}
}{% if KOMA class
  \KOMAoptions{parskip=half}}
\makeatother
\usepackage{xcolor}
\IfFileExists{xurl.sty}{\usepackage{xurl}}{} % add URL line breaks if available
\IfFileExists{bookmark.sty}{\usepackage{bookmark}}{\usepackage{hyperref}}
\hypersetup{
  pdftitle={Gestion de Portefeuille},
  pdfauthor={Patrick Hénaff},
  hidelinks,
  pdfcreator={LaTeX via pandoc}}
\urlstyle{same} % disable monospaced font for URLs
\usepackage[margin=1in]{geometry}
\usepackage{graphicx}
\makeatletter
\def\maxwidth{\ifdim\Gin@nat@width>\linewidth\linewidth\else\Gin@nat@width\fi}
\def\maxheight{\ifdim\Gin@nat@height>\textheight\textheight\else\Gin@nat@height\fi}
\makeatother
% Scale images if necessary, so that they will not overflow the page
% margins by default, and it is still possible to overwrite the defaults
% using explicit options in \includegraphics[width, height, ...]{}
\setkeys{Gin}{width=\maxwidth,height=\maxheight,keepaspectratio}
% Set default figure placement to htbp
\makeatletter
\def\fps@figure{htbp}
\makeatother
\setlength{\emergencystretch}{3em} % prevent overfull lines
\providecommand{\tightlist}{%
  \setlength{\itemsep}{0pt}\setlength{\parskip}{0pt}}
\setcounter{secnumdepth}{-\maxdimen} % remove section numbering

\linespread{1.05}
\usepackage[utf8]{inputenc}
\ifluatex
  \usepackage{selnolig}  % disable illegal ligatures
\fi
\newlength{\cslhangindent}
\setlength{\cslhangindent}{1.5em}
\newlength{\csllabelwidth}
\setlength{\csllabelwidth}{3em}
\newenvironment{CSLReferences}[2] % #1 hanging-ident, #2 entry spacing
 {% don't indent paragraphs
  \setlength{\parindent}{0pt}
  % turn on hanging indent if param 1 is 1
  \ifodd #1 \everypar{\setlength{\hangindent}{\cslhangindent}}\ignorespaces\fi
  % set entry spacing
  \ifnum #2 > 0
  \setlength{\parskip}{#2\baselineskip}
  \fi
 }%
 {}
\usepackage{calc}
\newcommand{\CSLBlock}[1]{#1\hfill\break}
\newcommand{\CSLLeftMargin}[1]{\parbox[t]{\csllabelwidth}{#1}}
\newcommand{\CSLRightInline}[1]{\parbox[t]{\linewidth - \csllabelwidth}{#1}\break}
\newcommand{\CSLIndent}[1]{\hspace{\cslhangindent}#1}

\title{Gestion de Portefeuille}
\author{Patrick Hénaff}
\date{Version: 26 déc. 2021}

\begin{document}
\maketitle

Les conditions de marché actuelles rendent particulièrement pertinentes
les méthodes quantitatives de gestion de portefeuille. Dans le contexte
Français, la baisse des taux met en cause la viabilité des fonds en
euros des contrats d'assurance-vie, un des piliers de l'épargne des
Français. Un objectif de ce cours sera de montrer comment la gestion
quantitative peut apporter un élément de réponse à ce problème.

Ce cours présente un panorama de la théorie et de la pratique de gestion
quantitative de portefeuille. On abordera la gestion d'un portefeuille
d'actions, et aussi, plus brièvement, la gestion obligataire
quantitative.

\hypertarget{manuel}{%
\section{Manuel}\label{manuel}}

Le cours utilise le manuel de Bernhard Pfaff ``Financial Risk Modeling
and Portfolio Optimization with R,'' 2ème édition (Pfaff, 2016). Le
manuel est disponible, en autres, sur Amazon.

\hypertarget{organisation-pratique}{%
\section{Organisation pratique}\label{organisation-pratique}}

Des incertitudes planent sur l'organisation du cours. Puisqu'il risque
de devoir se tenir ``à distance,'' un certain nombre de dispositions
pratiques doivent être mises en place.

Le principe général du cours est celui de la ``classe inversée.'' Il y a
8 modules dans le cours, et chacun s'articule selon le même schéma:

\begin{itemize}
\item
  avant le cours, chacun étudie les documents mis à disposition
  (articles, chapitre du manuel)
\item
  le module commence par une intervention, pour résumer le sujet et
  répondre aux questions,
\item
  en groupe de 2 ou 3, les étudiants réalisent ensuite les travaux
  pratiques propres à chaque module, sous forme de notebooks
  ``Rmarkdown.'' Tous les documents nécessaires se trouveront sur GitHub
  en temps utile, dans le dépot public \texttt{phenaff/IMT-GestionPort}.
\item
  après chaque journée , des vidéo-conférences seront programmées pour
  répondre aux questions soulevées par les travaux de groupe.
\end{itemize}

\hypertarget{evaluation}{%
\section{Evaluation}\label{evaluation}}

Chaque groupe choisit de rendre 4 TP parmi les 8 proposés, chaque TP
comptant pour 25\% de la note, qui sera commune à tous les membres du
groupe. Les TP sont à réaliser en notebook ``Rmarkdown'' (Xie et al.,
2019) et à rendre au format .pdf. Vous rendrez également le code source
.rmd. ``Rmarkdown'' est une technologie très utile à maîtriser, car elle
permet de produire des analyses \emph{reproductibles}, avec une mise en
page de grande qualité. La présentation et mise en page des documents
devra donc être soignée, et sera prise en compte dans l'évaluation. Les
TP sont à rendre 15 jours après le module correspondant. Vous êtes
fortement encouragés à profiter des vidéo-conférences pour valider
l'avancement de vos travaux de groupe.

\hypertarget{roles-dans-chaque-groupe-de-travail}{%
\section{Roles dans chaque groupe de
travail}\label{roles-dans-chaque-groupe-de-travail}}

Pour chaque TP donnant lieu à une évaluation, chaque groupe désigne un
``maître du temps'' chargé de s'assurer que le travail soit rendu en
temps et en heure. Indiquer clairement le nom de cette personne en
entête de chaque TP rendu.

\hypertarget{objectifs-du-cours}{%
\section{Objectifs du cours}\label{objectifs-du-cours}}

\begin{enumerate}
\def\labelenumi{\arabic{enumi}.}
\item
  Approfondir les propriétés statistiques des séries chronologiques
  financières.
\item
  Maitriser le modèle classique ``moyenne-variance'' de
  Markowitz(Markowitz, 1952), et comprendre ses limites. Savoir le
  mettre en œuvre et analyser les résultats. Appréhender comment le
  modèle de Black-Litterman répond à certaines limitations du modèle de
  Markowitz.
\item
  Comprendre l'apport de l'approche factorielle en gestion de
  portefeuille.
\item
  Appréhender les nouvelles approches de gestion fondées sur le ``risk
  budgeting.''
\item
  Réaliser un rapide survol des méthodes de gestion de portefeuille
  obligataire: couverture en sensibilité et adossement des flux.
\end{enumerate}

\hypertarget{logiciel}{%
\section{Logiciel}\label{logiciel}}

A chaque séance, on utilisera le logiciel R/Rstudio/Rmarkdown pour
mettre immédiatement en pratique les concepts présentés. Ce logiciel est
devenu un outil incontournable en finance quantitative, et en
particulier en gestion de portefeuille.

\hypertarget{avant-la-premiuxe8re-suxe9ance}{%
\section{Avant la première
séance}\label{avant-la-premiuxe8re-suxe9ance}}

\begin{itemize}
\item
  Si ce n'est pas le cas, se familiariser avec le système de gestion de
  version Git et Github. Installer un outil de gestion de version tel
  que SmartGit. Je suggère à chaque groupe de travail de créer un dépot
  privé sur GitHub.
\item
  installez R, RStudio, Rmarkdown, TinyTex et vérifiez que votre
  installation est opérationnelle en exécutant le document
  \texttt{TP-1/time.series.demo.Rmd}. Vérifiez votre maîtrise de R en
  faisant les exercices proposés dans ce document.
\end{itemize}

\hypertarget{programme}{%
\section{Programme}\label{programme}}

\textbf{Avant} chaque module, il est indispensable d'étudier les
documents fournis.

\hypertarget{module-1-152-suxe9ries-chronologiques-financiuxe8res-cont2001}{%
\subsection{Module 1 (15/2): Séries chronologiques financières (Cont,
2001)}\label{module-1-152-suxe9ries-chronologiques-financiuxe8res-cont2001}}

Dans cette séance introductive, on passera en revue les ``faits
stylisés'' caractéristiques des séries chronologiques financières, et
les méthodes de calcul de la covariance entre les actifs financiers.

Documents à lire avant le cours:

\begin{itemize}
\tightlist
\item
  Article de R. Cont (2001)
\item
  Note de cours ``conditional probability''
\item
  Chapitre 3 de Pfaff (2016)
\end{itemize}

Documents:

\begin{itemize}
\tightlist
\item
  slides-1.pdf
\end{itemize}

TP 1 (à rendre pour le 2022-03-02):

\begin{itemize}
\tightlist
\item
  Observation des faits stylisés.
\item
  Estimation de quelques distributions et modèles dynamiques.
\item
  Estimation de la corrélation entre séries.
\end{itemize}

\hypertarget{module-2-152-la-thuxe9orie-classique-markowitz1952}{%
\subsection{Module 2 (15/2): La théorie classique (Markowitz,
1952)}\label{module-2-152-la-thuxe9orie-classique-markowitz1952}}

On considère ici le travail d'Harry Markowitz, qui établit les
fondements de la gestion quantitative. Ce modèle reste important car il
a défini le vocabulaire et les concepts de base qui sont toujours
d'actualité.

Documents à lire avant le cours:

\begin{itemize}
\tightlist
\item
  Article de Markowitz (1952)
\item
  Note de cours
\item
  Chapitre 5 de Pfaff (2016)
\end{itemize}

Documents:

\begin{itemize}
\tightlist
\item
  slides-2.pdf
\item
  Notes-MV.pdf
\end{itemize}

TP 2 (à rendre pour le 2022-03-02):

\begin{itemize}
\tightlist
\item
  Construction d'une frontière efficiente.
\item
  Construction d'un portefeuille optimal moyenne/variance.
\end{itemize}

\hypertarget{module-3-162-medaf-moduxe8le-uxe0-un-facteur-et-mesure-de-performance.-moduxe8le-de-treynor-black.-distinction-gestion-activegestion-passive.-treynor1973.}{%
\subsection{Module 3 (16/2: MEDAF, modèle à un facteur et mesure de
performance. Modèle de Treynor-Black. Distinction ``gestion
active/gestion passive.'' (Treynor \& Black,
1973).}\label{module-3-162-medaf-moduxe8le-uxe0-un-facteur-et-mesure-de-performance.-moduxe8le-de-treynor-black.-distinction-gestion-activegestion-passive.-treynor1973.}}

Le modèle MEDAF (CAPM) et son pendant empirique, le modèle de marché à
un facteur sont tous les deux dûs à W. Sharpe. Ces modèles sont toujours
importants aujourd'hui car ils servent de base aux mesures de
performance des portefeuilles et des stratégies d'investissement. Dérivé
lui aussi des travaux de Markowitz, le modèle de Treynor-Black est aussi
une avancée importante, car il est à l'origine de la distinction
``gestion active/gestion passive.'' Ce sont néanmoins des modèles
fragiles, on les étudie aujourd'hui plus à cause du vocabulaire qu'ils
ont introduit que pour leur utilité pratique.

Documents à lire avant le cours:

\begin{itemize}
\tightlist
\item
  Article de Treynor \& Black (1973)
\item
  Chapitre 5 de Pfaff (2016)
\item
  Notes sur les mesures de performance
\end{itemize}

Documents:

\begin{itemize}
\tightlist
\item
  slides-MEDAF.pdf
\item
  Notes-CAPM.pdf
\end{itemize}

TP 3 (à rendre pour le 2022-03-03):

\begin{itemize}
\tightlist
\item
  Construction d'un modèle à 1 facteur
\item
  Optimisation de portefeuille selon Treynor-Black
\end{itemize}

\hypertarget{module-4-162-moduxe8le-de-treynor-black-suite.-risque-de-moduxe9lisation.-boyle2012-ste1997}{%
\subsection{Module 4 (16/2): Modèle de Treynor Black (suite). Risque de
modélisation. (Boyle et al., 2012; Stevens,
1997)}\label{module-4-162-moduxe8le-de-treynor-black-suite.-risque-de-moduxe9lisation.-boyle2012-ste1997}}

Identification du ``risque de modélisation'' dans le cadre du modèle
moyenne/variance, et en particulier du risque lié à l'utilisation de la
matrice de covariance.

Documents à lire avant le cours:

\begin{itemize}
\tightlist
\item
  Articles cités
\item
  Chapitre 10 de Pfaff (2016)
\end{itemize}

Documents:

\begin{itemize}
\tightlist
\item
  slides-MVO.pdf
\end{itemize}

TP 4 (à rendre pour le 2022-03-03):

\begin{itemize}
\tightlist
\item
  Impact de la matrice de covariance sur les résultats de modèle
  moyenne-variance.
\end{itemize}

\hypertarget{module-5-23-moduxe8le-de-black-litterman.-he2005}{%
\subsection{Module 5 (2/3): Modèle de Black-Litterman. (He \& Litterman,
2005)}\label{module-5-23-moduxe8le-de-black-litterman.-he2005}}

Le modèle de Black-Litterman et ses nombreuses extensions est très prisé
des gestionnaires du fait de sa flexibilité. Il permet également de
limiter les risques de modélisation identifiés précédemment.

Documents à lire avant le cours:

\begin{itemize}
\tightlist
\item
  Chapitre 13.3 de Pfaff (2016)
\item
  Article de Litterman et He
\end{itemize}

Documents:

\begin{itemize}
\tightlist
\item
  slides-BL.pdf
\item
  Notes-BL.pdf
\end{itemize}

TP 5 (à rendre pour le 2022-03-17):

\begin{itemize}
\tightlist
\item
  Comparaison du modèle M/V et du modèle de Black-Litterman.
\end{itemize}

\hypertarget{module-6-23-approche-factorielle-fama1992-fama1993-harvey2016a}{%
\subsection{Module 6 (2/3): Approche factorielle (E. F. Fama \& French,
1992; F. Fama \& French, 1993; Harvey et al.,
2016)}\label{module-6-23-approche-factorielle-fama1992-fama1993-harvey2016a}}

\begin{itemize}
\tightlist
\item
  Définition et identification des facteurs
\item
  Estimation et limites statistiques
\item
  Modèles d'allocation factoriels
\end{itemize}

Documents à lire avant le cours:

\begin{itemize}
\tightlist
\item
  Article cités
\end{itemize}

Documents:

\begin{itemize}
\tightlist
\item
  slides-MF.pdf
\end{itemize}

Pas de TP.

\hypertarget{module-7-153-muxe9thodes-ruxe9centes-de-gestion-de-portefeuille-risk-budgeting-bruder2012}{%
\subsection{Module 7 (15/3): Méthodes récentes de gestion de
portefeuille, ``risk budgeting'' (Bruder \& Roncalli,
2012)}\label{module-7-153-muxe9thodes-ruxe9centes-de-gestion-de-portefeuille-risk-budgeting-bruder2012}}

\begin{itemize}
\tightlist
\item
  Modèle 1/N
\item
  Modèle ``risk parity''
\end{itemize}

Documents à lire avant le cours:

\begin{itemize}
\tightlist
\item
  Chapitre 11 de Pfaff (2016)
\item
  Article de Bruder et Roncalli
\end{itemize}

Documents:

\begin{itemize}
\tightlist
\item
  slides-RB.pdf
\end{itemize}

TP 7 (à rendre pour le 2022-03-30):

\begin{itemize}
\tightlist
\item
  calcul de portefeuilles selon des méthodes de ``risk budgeting''
\end{itemize}

\hypertarget{module-8-153-gestion-de-portefeuille-obligataire-immunisation-et-duxe9dication-henaff2012.}{%
\subsection{Module 8 (15/3): Gestion de portefeuille obligataire:
Immunisation et dédication (Hénaff,
2012).}\label{module-8-153-gestion-de-portefeuille-obligataire-immunisation-et-duxe9dication-henaff2012.}}

Survol du problème de gestion obligataire et des approches classiques:
couverture en sensibilité et adossement des flux.

Documents à lire avant le cours:

\begin{itemize}
\tightlist
\item
  Chapitre 6 ``Fixed Income Risk Management''
\end{itemize}

TP 8 (à rendre pour le 2022-03-30):

\begin{itemize}
\tightlist
\item
  Construction d'un portefeuille obligataire par programmation linéaire.
\end{itemize}

\hypertarget{bibliographie}{%
\section*{Bibliographie}\label{bibliographie}}
\addcontentsline{toc}{section}{Bibliographie}

\hypertarget{refs}{}
\begin{CSLReferences}{1}{0}
\leavevmode\hypertarget{ref-Boyle2012}{}%
Boyle, P., Garlappi, L., Uppal, R., \& Wang, T. (2012). {Keynes meets
Markowitz: The trade-off between familiarity and diversification}.
\emph{Management Science}, \emph{58}(2), 253--272.
\url{https://doi.org/10.1287/mnsc.1110.1349}

\leavevmode\hypertarget{ref-Bruder2012}{}%
Bruder, B., \& Roncalli, T. (2012). {Managing Risk Exposures Using the
Risk Budgeting Approach}. \emph{SSRN Electronic Journal}, 1--33.
\url{https://doi.org/10.2139/ssrn.2009778}

\leavevmode\hypertarget{ref-Cont2001}{}%
Cont, R. (2001). {Empirical properties of asset returns: stylized facts
and statistical issues}. \emph{Quantitative Finance}, \emph{1},
223--236.
\url{http://citeseerx.ist.psu.edu/viewdoc/summary?doi=10.1.1.16.5992}

\leavevmode\hypertarget{ref-Fama1992}{}%
Fama, E. F., \& French, K. R. (1992). {The Cross-Section of Expected
Stock Returns}. \emph{Journal of Finance}, \emph{1}, 427--465.

\leavevmode\hypertarget{ref-Fama1993}{}%
Fama, F., \& French, R. (1993). {Common risk factors in the returns on
stocks and bonds}. \emph{Journal of Financial Economics}, \emph{33},
3--56. \url{https://doi.org/10.1016/0304-405X(93)90023-5}

\leavevmode\hypertarget{ref-Harvey2016a}{}%
Harvey, C. R., Liu, Y., \& Zhu, H. (2016){... and the Cross-Section of
Expected Returns}. \emph{Review of Financial Studies}, \emph{29}(1),
5--68. \url{https://doi.org/10.1093/rfs/hhv059}

\leavevmode\hypertarget{ref-He2005}{}%
He, G., \& Litterman, R. (2005). \emph{{The Intuition Behind
Black-Litterman Model Portfolios}}.
\url{https://doi.org/10.2139/ssrn.334304}

\leavevmode\hypertarget{ref-Henaff2012}{}%
Hénaff, P. (2012). \emph{{Topics in Empirical Finance with R and
Rmetrics}}.

\leavevmode\hypertarget{ref-Markowitz1952}{}%
Markowitz, H. M. (1952). {Portfolio Selection}. \emph{The Journal of
Finance}, \emph{7}(1), 77--91.

\leavevmode\hypertarget{ref-Pfaff2016}{}%
Pfaff, B. (2016). \emph{{Financial Risk Modelling and Portfolio
Optimization with R}} (p. 634). John Wiley \& Sons, Ltd.

\leavevmode\hypertarget{ref-Ste1997}{}%
Stevens, G. V. G. (1997). \emph{{On the Inverse of the Covariance Matrix
in Portfolio Analysis}} (No. 587).

\leavevmode\hypertarget{ref-Treynor1973}{}%
Treynor, J. L., \& Black, F. (1973). {How to Use Security Analysis to
Improve Portfolio Selection}. \emph{The Journal of Business},
\emph{46}(1), 66--86. \url{http://www.jstor.org/stable/2351280}

\leavevmode\hypertarget{ref-Wurtz2015}{}%
Würtz, D., Setz, T., Chalabi, Y., \& Chen, W. (2015). \emph{{Portfolio
Optimization with R/Rmetrics}}.
\url{https://www.rmetrics.org/downloads/9783906041018-fPortfolio.pdf}

\leavevmode\hypertarget{ref-Xie2019}{}%
Xie, Y., Allaire, J. J., \& Grolemund, G. (2019). \emph{{R Markdown: The
Definitive Guide}}.

\end{CSLReferences}

\end{document}
